\documentclass[12pt]{article}

\usepackage[utf8]{inputenc}
\usepackage[brazil]{babel}
\usepackage{amsmath,amssymb}
\usepackage{pdfsync}
\usepackage[all]{xy}
\usepackage{color}

\newcommand{\resposta}[1]{ \noindent {\bf Solução}.{\color{blue} #1}}

\title{{\large Universidade de Brasília \\ Instituto de Ciências Exatas \\
Departamento de Ciência da Computação} \\[1cm]
CIC 117536 - Projeto e Análise de Algoritmos \\[.5cm]  Terceira Prova \\[.5cm] Turma: B}
\author{{\bf NP-completude}}
\date{Prof. Flávio L. C. de Moura \\[.5cm] \today}

\begin{document}
\maketitle

\begin{enumerate}
\item {\bf (2.5 pontos)} O problema 2-SAT tem como instâncias as
  fórmulas lógicas formadas por conjunções de disjunções de até dois
  literais, onde um literal é uma variável booleana ou a negação de
  uma variável booleana. Por exemplo, a expressão a seguir é uma
  instância de 2-SAT:

  $$(x_1\lor \neg x_2)\land (\neg x_1 \lor \neg x_3) \land (x_1 \lor x_2) \land x_3$$

  Prove que 2-SAT $\in$ P.

 
  \resposta{
    Cria-se um digrafo contendo cada variável e a negação de cada uma em cada vértice e para cada ${(x\lor y)}$ é feito um vértice ligando ${\neg x}$ a y e um vértice ligando ${\neg y}$ a x.

	Com o digrafo pronto representando a fórmula, é feita a identificação de componentes fortemente conectados sem que a variável e a negação dela estejam no mesmo componente, isso é feito em tempo linear.

	Por essa montagem do digrafo, se houvesse um caminho de uma variável até a a negação dela, também teria um caminho da negação a ela mesma.

	Com isso, é possível verificar um caminho de uma variável à negação verificando se estão em componentes conectados em tempo polinomial.
  }
  
\item {\bf (2.5 pontos)} Em aula, assumimos que SAT é um problema
  NP-completo (Teorema de Cook-Levin), e a partir deste fato mostramos
  que 3-SAT e CLIQUE também são problemas NP-completos. As reduções
  foram feitas de acordo com o seguinte diagrama:

  $$\xymatrix{
    SAT \ar[d] \\
    3\mbox{-}SAT \ar[d] \\
    CLIQUE 
  }$$
  
  Um ciclo Hamiltoniano é um ciclo simple que visita cada vértice de
  um grafo exatamente uma vez. Considere o problema de decisão
  HAM-CYCLE que pergunta se um dado grafo (não-dirigido) $G$ possui um
  ciclo Hamiltoniano. Mostre que HAM-CYCLE é um problema
  NP-completo. Sua solução deve ser construída a partir de SAT, 3-SAT
  ou CLIQUE. Caso, você não veja como reduzir diretamente HAM-CYCLE a
  partir destes, mas sabe como fazê-lo a partir de um certo problema
  $Q$ então inicialmente mostre que $Q$ é NP-completo a partir de SAT,
  3-SAT ou CLIQUE, e assim por diante. Digamos que você não saiba como
  mostrar que $Q$ é NP-completo diretamente a partir de SAT, 3-SAT ou
  CLIQUE, mas você sabe como fazê-lo a partir de outro problema $Q'$,
  e também sabe como mostrar que $Q'$ é NP-completo a partir de 3-SAT,
  por exemplo. Então o diagrama correspondente à sua solução seria:

$$\xymatrix{
  SAT \ar[d] & & & \\
  3\mbox{-}SAT \ar[d]\ar[r] & Q' \ar[r] & Q \ar[r] & \mbox{HAM-CYCLE}  \\
  CLIQUE & & & 
}$$

E todas as reduções (de 3-SAT para $Q'$, de $Q'$ para $Q$ e de $Q$ para HAM-CYCLE) devem ser detalhadas na sua solução.

\resposta{
    Para provar que o ham-cycle não-dirigido é um problema np completo, pode-se mudá-lo para um ham-cycle dirigido para provar. Contudo, é necessário provar que o ham-cycle dirigido é um problema np completo. Para isso, usa-se a redução do 3-SAT.

Primeiro, para verificar que o ham-cyle dirigido é NP, basta usar um algoritmo que verifica se cada vértice que faz parte do ciclo aparece uma única vez no ciclo e percorrendo o caminho possa retornar ao primeiro vértice ao final do ciclo. Isso pode ser feito em tempo polinomial. Após isso, pode-se verificar se é NP completo a partir do 3-SAT.

Um 3-SAT contendo n variáveis possui ${2^n}$ possíveis cláusulas. Um grafo correspondendo a essas possibilidades é construído com n caminhos representando as variáveis e cada caminho possui 2k nós, ligados por arestas, em que k é o número de cláusulas da fórmula.

Depois é preciso ligar os caminhos e isso pode ser feito conectando cada vértice que inicia um caminho ao início e ao fim do próximo caminho e o mesmo é feito para o nó que termina um caminho, ligando ao ao início e ao fim do próximo caminho. Por fim, acrescenta-se 2 vértices intermediários, o primeiro com uma aresta em direção ao vértice de início do primeiro caminho e com outra aresta em direção ao vértice que termina o primeiro caminho. A partir do vértice que inicia o último caminho, é feita uma aresta em direção ao segundo vértice intermediário e, a partir do vértice que finaliza esse caminho, uma outra aresta é feita em direção a esse vértice intermediário. Por fim, uma aresta é feita partindo do segundo nó intermediário ao primeiro.

Em seguida são adicionados vértices representando as cláusulas, cada cláusula será ligada a 2 nós de cada caminho correspondente a uma variável existente na cláusula, os nós a que serão ligados corresponde ao primeiro nó ímpar do caminho e o primeiro nó par que ainda não foram ligados a uma cláusula (considera-se que os nós de cada caminho estão numerados de 1 a 2k, a partir do nó que inicia o caminho). Se na cláusula, a variável não-negada está contida, o caminho é feito a partir do nó ímpar até o nó par passando pelo nó da cláusula. Caso seja a negação da variável que esteja presente o caminho é contrário, sai do nó par em direção ao nó ímpar passando pelo nó da cláusula.

Nesse grafo cada um dos n caminhos podem ser percorridos da direita para a esquerda ou da esquerda para a direita. Desse modo existem ${2^n}$ ciclos diferentes. Além disso, atravessando um caminho da esquerda para a direita permite verificar que uma variável é verdadeira e da direita para a esquerda que é falsa e cada cláusula é visitada passando por algum caminho na direção correta para cada variável presente na cláusula. Assim, o ciclo Hamiltoniano dirigido satisfaz o 3-SAT e pode ser considerado NP-completo.

Para verificar que o ciclo não-dirigido também é NP-completo, basta substituir cada aresta do não-dirigido por 2 arestas em sentidos contrários. Isso mantém as características de um não-dirigido, mas o torna dirigido. Essa mudança pode ser feita em tempo polinomial, já que é apenas um loop que passa por todas as arestas e as substitui por outras duas. Desse modo, como a redução é possível, o ham-cycle não-dirigido é também NP-completo.


  }
  
\item {\bf (2.5 pontos)} Considere o seguinte jogo em um grafo
  (não-dirigido) $G$, que inicialmente contém 0 ou mais bolas de gude
  em seus vértices: um movimento deste jogo consiste em remover duas
  bolas de gude de um vértice $v\in G$, e adicionar uma bola a algum
  vértice adjacente de $v$. Agora, considere o seguinte problema: Dado
  um grafo $G$, e uma função $p(v)$ que retorna o número de bolas de
  gude no vértice $v$, existe uma sequência de movimentos que remove
  todas as bolas de $G$, exceto uma? Mostre que este problema é
  NP-completo. A mesma observação feita no exercício anterior vale
  aqui: a prova deve ser feita a partir de problemas que provamos
  serem NP-completos, e reduções intermediárias, caso existam, devem
  ser incluídas na solução.

  \resposta{
    Para verificar que o problema é NP-completo, primeiro verifica se é NP. Isso pode ser mostrado percorrendo os vértices e para cada vértice que possuir 2 ou mais bolas de gude realizar a remoção de todas as bolas possíveis de duas em duas e colocar a quantidade necessária de bolas no próximo vértice, o percorrimento é feito em um único sentido (direita para esquerda ou esquerda para direita), passando por todos os vértices uma vez apenas antes de voltar ao primeiro vértice. Esse percorrimento pode ser feito em tempo polinomial.

	Pode-se mostrar que é possível construir ham-cycles a partir desse problema, já que, como foi mostrado que o ham-cycle é um np completo, a redução correta do ham-cycle mostra que esse problema também é np completo.
	
	Assim, constroi-se um grafo que soluciona esse problema de forma que um dos nós possui 2 bolas de gude e todos os outros possuem uma bola de gude. Assim, no vértice que possui 2 bolas de gude, retira-se as duas e adiciona um ao próximo vértice. Depois, avança para o próximo vértice, que passa a ter 2 bolas, faz o mesmo procedimento e avança até chegar ao último vértice. Nesse nó, o mesmo procedimento é feito e 2 bolas são removidas, ficando com nenhuma nele e acrescentando uma ao próximo nó que foi o que iniciou. Assim, ao final, percorre-se uma única vez cada vértice formando um ciclo e sobra apenas 1 bola no grafo. Desse modo, mostra-se que o ham-cycle pode ser reduzido a esse problema e, como é np e np-hard, esse problema é np-completo.
	
  }
  
\item {\bf (2.5 pontos)} Uma fórmula booleana em {\it forma normal conjuntiva com disjunção exclusiva (FNCX)} é uma conjunção de diversas cláusulas, e cada cláusula é uma disjunção exclusiva (XOR) de diversos literais. Lembre-se que a disjunção exclusiva é dada por:

  $$\begin{array}{|l|l|l|}\hline
      a & b & a \oplus b \\ \hline
      V & V & F \\ \hline
      V & F & V \\ \hline
      F & V & V \\ \hline
      F & F & F \\ \hline
  \end{array}$$

  O problema FNCX-SAT pergunta se uma dada fórmula em FNCX é
  satisfatível. Mostre que o problema FNCX-SAT está em $P$, ou então
  que FNCX-SAT é NP-completo. No último caso, a mesma observação feita
  nos dois exercícios anteriores vale aqui: a prova deve ser feita a
  partir de problemas que provamos serem NP-completos, e reduções
  intermediárias, caso existam, devem ser incluídas na solução.

  \resposta{
    Para provar que o FNCX-SAT é polinomial, pode-se verificar que cláusulas de disjunção exclusiva são equivalentes a equações lineares em que os valores possíveis são 0 ou 1. Isso ocorre pela característica de que 0+1=1, 1+0=1, 1+1=0 e 0+0=0 e pela tabela verdade é o mesmo que acontece com uma cláusula de disjunção exclusiva. Desse modo, uma cláusula pode ser escrita como uma operação aritmética de soma das variáveis e se a cláusula for satisfeita o resultado é 1. Como a negação não possui uma representação direta em um sistema de equações lineares, pode-se alterá-la de modo que seja substituída pela variável. Isso pode ser feito com ${ x \oplus 1 = \neg x}$, já que, se x=1, essa operação resulta em 0 e, se x=0, resulta em 1 que é exatamente a negação de x. Desse modo, para representar uma cláusula como uma equação linear, basta colocar como uma soma de todas as variáveis presentes nela e somar 1 para cada negação que existe na cláusula.

	Em um sistema de equações lineares o resultado depende de cada equação presente. Da mesma forma, na fórmula do FNCX-SAT, o resultado depende de cada cláusula devido à conjunção. Assim, é possível representar como um sistema de equações lineares, que se representado em forma matricial possui 1s para cada variável presente e 0s para cada variável ausente na cláusula e o resultado como 0 ou 1, sendo que cada cláusula corresponde a uma linha.

	Como é um sistema de equações lineares, pode ser resolvido pelo método de eliminação de Gauss que é de tempo polinomial (${O(n^3)}$). Logo, FNCX-SAT pertence a P.
}
\end{enumerate}

\end{document}

%%% Local Variables:
%%% mode: latex
%%% TeX-master: t
%%% End:
